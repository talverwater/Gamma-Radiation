\documentclass[]{article}
\usepackage[letterpaper, margin=1in]{geometry}
\twocolumn
\usepackage{stfloats}
%opening
\title{Gamma Radiation preparation report}
\author{Tal Silverwater}
\date{}

\begin{document}

\maketitle


\begin{table*}[!b]
	\centering
	\begin{tabular}{||c | c c c c c||} 
		\hline
		Nucleon & Mass & Spin & Magnetic Moment & Radius & Charge \\ [0.5ex] 
		\hline
		Proton & $1.00727647 u$ & $\frac{1}{2}\hbar$ & $2.7928474*\frac{e\hbar}{2m_p}$ & $10^{-15}m$ & $1e$ \\ 
		\hline
		Neutron & $1.00866490 u$ & $\frac{1}{2}\hbar$ & $-1.913043*\frac{e\hbar}{2m_p}$ & $10^{-15}m$ & 0  \\
		\hline
		
	\end{tabular}
	\caption{This table explains the property's of protons and neutrons}
\end{table*}

\section{General subjects in nuclear \\ physics}
\subsection{Typical sizes}

The typical radius of a nucleus is: $$R=1*10^{-15}m=1fermi=1fm$$
The typical radius of a atom is: $$R=1*10^{-10}m=1\AA$$
%can add more like forces and time scales

\subsection{Nuclear structure}

Rutherford has found in his experiments that the atom has a dense core in the middle (the nucleus), surrounded by a cloud of electrons.
The size of the nucleus for an aluminum atom is: $$r_{nucleus}=3*10^{-15}m$$ (found by Rutherford).
Since Rutherford experiments showed that the radius of the nucleus is: $$R=r_{0}A^{-\frac{1}{3}}$$ where A is the atomic mass number (number of protons times number of neutrons) and: $$r_0=1.2*10^{-15}m=1.2fm$$

\subsection{Components of the nuclues and their property's}

The nucleus is made out of protons and neutrons. These particles are called hadrons and they are composed from Quarks. The protons are made of 2 up quarks and one down quark and the neutron is made out of two down quarks and one up quark.The property's of these particles are detailed in table 1.


\subsection{Forces in the nucleus}

There are two main forces inside the atomic nucleus:

1) The electromagnetic force, that acts between two charged particles according to: $$F_{12}=\frac{kq_1q_2}{|\vec{r}|^2}\hat{r}$$ This force is a repellent force in the nucleus because all the protons are positively charged, and therefore works in order to disassemble the nucleus.

2) The strong force, which is a force pulling all the parts of the nucleus together. This force is short distance which leads to a finite nucleus. It is much stronger then the coulomb force in short distances (for two protons with a distance of 2 fm the coulomb force is 60 N and the strong force is $2*10^{3} N$), but if the distance is really short the strong force is repulsive. This force is independent of charge, and dependent on spin. For equations and explanation see Ohanian Modern Physics page 351 (might be added later).

\subsection{The Liquid-Drop model}

This model is based on the similarity of intermolecular forces and the strong force, suggesting the nucleus acts like a liquid (can't be a solid because of zero point energy). In that view the nucleus is an almost incompressible fluid with constant density. This model leads to an approximated formula for the binding energy of the nuclei that depends on A and Z (atomic mass and number of protons). The binding energy term is: $$a_1A-a_2A^{2/3}$$ where $a_1$ and $a_2$ is a positive constant. The electrostatic potential can be written as: $$-a_3\frac{Z^2}{A^{1/3}}$$ where $a_3$ is an adjustable constant. The quantum mechanical correction gives: $$-a_4\frac{(A-2Z)^2}{A}$$ which is called the asymmetry energy. In total the binding energy B is: $$B=a_1A-a_2A^{2/3}-a_3\frac{Z^2}{A^{1/3}}-a_4\frac{(A-2Z)^2}{A}$$ From experiments we find: $$a_1=15.753Mev$$ $$a_2=17.804Mev$$ $$a_3=0.7103Mev$$ $$a_4=94.77Mev$$ This is called Weizs\"acker's semiempirical formula for binding energy. The formula for the mass is: $$M=Zm_H+(A-Z)m_n-\frac{B}{c^2}$$ where $m_H$ is equal to $Z(m_p+m_e)$ (you can plug in values and get eq 44 in page 357).
This approximation is good for large nuclei and is not expected to work for small A's. (Note- the larger B is the atom is more stable and when negitive the atom will burst apart). The nucleus is most stable when $\frac{\partial M}{\partial Z}=0$. Solving for Z we get: $$Z=\frac{A}{2+0.01499A^{2/3}}$$

\subsection{Basic idea of shell model}

The Liquid-Drop model isn't quantum and therefore misses important aspects of the nucleus,like the quantization of the angular momentum. Every nucleus has a spin which is a multiple of $\frac{1}{2}\hbar$. It also misses the energy levels of the nucleus, which is a potential sorce of $\gamma$ radiation.\\
The shell model attempts to calculate the the energy levels of the protons and neutrons in a nucleus under the assumption that the net force on every nucleon is approximated by an average central force. The idea is that nucleons can only be scattered near the nuclear surface (from the exclusion principal. more detailed in page 360). This means there is no force in the nucleus and very strong forces on the surface.

\section{Radioactivity}
\subsection{Statistics of radiation}

The statistics that describe radiation is given by the half-life formula: $$N(t)=N_0*(\frac{1}{2})^{t/\tau}$$ where $N(t)$ is the number of particles at time t, $N_0$ is the initial number of particles and $\tau$ is the half-life time.

\subsection{Background radiation}

This kind of radiation is ionizing, and is present in the environment. Background radiation originates from a variety of sources, both natural and artificial. Sources include cosmic radiation, and environmental radioactivity from such as naturally occurring radioactive materials including radon and radium, and fallout from nuclear weapons testing and nuclear accidents.

\section{$\alpha$ $\beta$ and $\gamma$ radiation}
\subsection{The type of radiation and its \\ emission process}
\subsubsection{$\alpha$ decay}

$\alpha$ particles are a nucleus of a helium 4 atom and is comprised of 2 protons and 2 neutrons. These particles are emitted when a massive and/or unstable particle break apart. 
There are 2 reasons why these particles are emitted (instead of protons, neutrons...):

1) Emission of $\alpha$ particles conserves the wave-function symmetry, which prevents a particle from spontaneously changing from exhibiting Bose–Einstein statistics (if it had an even number of nucleons) to Fermi–Dirac statistics (if it had an odd number of nucleons) or vice versa.

2) $\alpha$ particles have really high binding energy. This means it will not tend to break apart. the result is that heavy elements have enough energy in them to create this particle while other models of decay require additional energy. \\
The mechanism that explains $\alpha$ decay is quantum tunneling. In this process the $\alpha$ particle teleports out of the large nucleus to a point where the strong force is not strong enough to pull the $\alpha$ particle back, and then the electromagnetic force violently pushes the $\alpha$ particle away.

\subsubsection{$\beta$ decay}

$\beta$ particles are high energy electrons ($\beta^-$) or positrons ($\beta^+$). 






\end{document}
