\documentclass[]{article}
\usepackage[letterpaper, margin=1in]{geometry}
\twocolumn
\usepackage{stfloats}
%opening
\title{Gamma Radiation preparation report}
\author{Tal Silverwater}
\date{}

\begin{document}

\maketitle


\begin{table*}[!b]
	\centering
	\begin{tabular}{||c | c c c c c||} 
		\hline
		Nucleon & Mass & Spin & Magnetic Moment & Radius & Charge \\ [0.5ex] 
		\hline
		Proton & $1.00727647 u$ & $\frac{1}{2}\hbar$ & $2.7928474*\frac{e\hbar}{2m_p}$ & $10^{-15}m$ & $1e$ \\ 
		\hline
		Neutron & $1.00866490 u$ & $\frac{1}{2}\hbar$ & $-1.913043*\frac{e\hbar}{2m_p}$ & $10^{-15}m$ & 0  \\
		\hline
		
	\end{tabular}
	\caption{This table explains the property's of protons and neutrons}
\end{table*}

\section*{General subjects in nuclear physics}
\subsection*{Typical sizes}

The typical radius of a nucleus is: $$R=1*10^{-15}m=1fermi=1fm$$
The typical radius of a atom is: $$R=1*10^{-10}m=1\AA$$
%can add more like forces and time scales

\subsection*{Nuclear structure}

Rutherford has found in his experiments that the atom has a dense core in the middle (the nucleus), surrounded by a cloud of electrons.
The size of the nucleus for an aluminum atom is: $$r_{nucleus}=3*10^{-15}m$$ (found by Rutherford).
Since Rutherford experiments showed that the radius of the nucleus is: $$R=r_{0}A^{-\frac{1}{3}}$$ where A is the atomic mass number (number of protons times number of neutrons) and: $$r_0=1.2*10^{-15}m=1.2fm$$

\subsection*{Components of the nuclues and their property's}

The nucleus is made out of protons and neutrons. These particles are called hadrons and they are composed from Quarks. The protons are made of 2 up quarks and one down quark and the neutron is made out of two down quarks and one up quark.The property's of these particles are detailed in table 1:


\subsection*{\\ \\ Forces in the nucleus}

There are two main forces inside the atomic nucleus:

1) The electromagnetic force, that acts between two charged particles according to: $$F_{12}=\frac{kq_1q_2}{|\vec{r}|^2}\hat{r}$$ This force is a repellent force in the nucleus because all the protons are positively charged, and therefore works in order to disassemble the nucleus.

2) The strong force, which is a force pulling all the parts of the nucleus together. This force is short distance which leads to a finite nucleus. It is much stronger then the coulomb force in short distances (for two protons with a distance of 2 fm the coulomb force is 60 N and the strong force is $2*10^{3} N$), but if the distance is really short the strong force is repulsive. This force is independent of charge, and dependent on spin. For equations and explanation see Ohayon Modern Physics page 351 (might be added later).

\subsection*{The Liquid-Drop model}

This model is based on the similarity of intermolecular forces and the strong force, suggesting the nucleus acts like a liquid (can't be a solid because of zero point energy).






\end{document}
